\documentclass{article}



\usepackage{amsthm}
\usepackage{xcolor}
\usepackage{tikz-cd}
\usepackage{quiver}
\usepackage{mathtools}
\usepackage{bbm}
\usepackage[hidelinks]{hyperref}
\usepackage{stmaryrd}


\theoremstyle{definition}
\newtheorem{definition}{Definition}



\title{Capstone Summary}
\author{Daniel Sinderson}
\date{March 2024}

\begin{document}
\maketitle

% PREAMBLE AND INTRO
\section*{Introduction}
For my senior capstone project in mathematics I'm learning category theory,
specifically with an interest in recent work applying category theory to dynamical systems.

Category theory studies composition.
By studying and abstracting notions of composition, categories manage to encapsulate
a shockingly flexible and wide-reaching language for describing and working with structure of all sorts.
Much like the study of sets and the set membership of elements turns out to be capable of formalizing all of mathematics,
so too does category theory, except from a bird's eye point of view:
the details of a particular field go out of focus and only its high level structural patterns remain.
This is useful.
High levels of abstraction provide high levels of generality,
and a general, common language of how things are structured is useful for both organizing thought and sharing it.
This is the primary reason I'm drawn to category theory.

The second reason is its potential for opening a broader class of thinking and phenomena to mathematical rigor.
I have a Bachelor' degree in anthropology.
I still read 5-10 ethnographies and works on social theory every year,
and I'm consistently impressed by the qualitative depth and intellectual creativity of the field.
I'm also consistently left wanting more by the body of theory underpinning the research and how it's interpreted.
Not because it's not brilliant, but because of the extraordinary imprecision of its language,
and the opportunities that that affords for misunderstanding, misinterpretation, and misuse.
Part of my wandering into category theory then and its applications to systems theory
is a search for a better language of structure: a language that's flexible and abstract enough
to hold the complications and self-referentiality of human relations, and rigorous and precise enough
to do so clearly.
I've only just arrived, but the view is promising and, more importantly, its spectacular.


%WAIT, WHAT'S A CATEGORY? I DO NOT THINK THIS WORD MEANS WHAT YOU THINK IT MEANS.
\section*{Category Theory, Briefly}
The first step to understanding category theory is to understand what a category is.
If you're new to higher mathematics it's time to take a deep breath.
This will seem like a lot.

\begin{definition}[Category]
    A category $\mathcal{C}$ is defined by the following:
    \begin{enumerate}
        \item $\mathcal{C}$ contains a collection of objects $ob(\mathcal{C})$. We'll denote that an object is in a category using set notation: $c\in\mathcal{C}$.
        \item For any two objects $a$, $b$ $\in \mathcal{C}$ there is a collection of morphisms, or arrows, between those objects $\mathcal{C}(a,b)$ called the homset. This is short for ``set of homomorphism." We'll denote an element $f\in\mathcal{C}(a,b)$ using function notation: $f:a\rightarrow b$.
        \item Every object $a\in \mathcal{C}$ has a morphism to itself $id_a:a\rightarrow a$ called its identity. This morphism doesn't do anything. It's like multiplying a number by 1.
        \item For every two morphisms $f:a\rightarrow b$ and $g: b\rightarrow c$ there's a third morphism $g\circ f:a\rightarrow c$ that's their composition. The circle is the symbol for function composition and $g \circ f$ is read ``g after f."

              \[\begin{tikzcd}
                      a && b \\
                      \\
                      && c
                      \arrow["f", from=1-1, to=1-3]
                      \arrow["g", from=1-3, to=3-3]
                      \arrow["{g\circ f}"', from=1-1, to=3-3]
                  \end{tikzcd}\]

    \end{enumerate}
    These objects and morphisms are then under the following two constraints:
    \begin{enumerate}
        \item (Unitality) Any morphism $f:a\rightarrow b$ can be composed with the identity morphisms of $a$ and $b$ such that $f\circ 1_a=1_b\circ f=f$. This makes the following diagram commute. What this means is that both paths from $a$ to $b$ here are equivalent.

              \[\begin{tikzcd}
                      a && b \\
                      \\
                      a && b
                      \arrow["f", from=1-1, to=1-3]
                      \arrow["{1_a}"', from=1-1, to=3-1]
                      \arrow["{1_b}", from=1-3, to=3-3]
                      \arrow["f"', from=3-1, to=3-3]
                  \end{tikzcd}\]

        \item (Associativity) For any morphisms $f:a\rightarrow b$, $g:b\rightarrow c$, and $h:c\rightarrow d$, $h\circ (g\circ f)=(h\circ g)\circ f$. Since it doesn't matter what order we apply the morphisms, we write this $h\circ g \circ f$.

              \[\begin{tikzcd}
                      a & b & c & d
                      \arrow["f"', from=1-1, to=1-2]
                      \arrow["g"', from=1-2, to=1-3]
                      \arrow["h"', from=1-3, to=1-4]
                      \arrow["{{g\circ f}}", curve={height=-24pt}, from=1-1, to=1-3]
                      \arrow["{{h\circ g}}"', curve={height=24pt}, from=1-2, to=1-4]
                  \end{tikzcd}\]

    \end{enumerate}
\end{definition}

Let's break this down a little bit.
We have a collection of objects with connections between them, and a couple of rules.
We know that every object is connected to itself.
And we know that if object $a$ is connected to $b$, and $b$ is connected to $c$, then $a$ is connected to $c$ through $b$.\footnote{For people with some background in math or logic this might be familiar as the transitive property.}
That's it.
Categories are honestly pretty simple.
But from these humble beginnings we're going to climb very high.




\end{document}



